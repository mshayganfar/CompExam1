\documentclass[11pt]{article}

\usepackage{graphicx}
%\usepackage{algorithmic}
%\usepackage{algorithm}
\usepackage{amssymb}
\usepackage{amsmath}
\usepackage{enumerate}
\usepackage[mathscr]{euscript}

\begin{document}

\pagenumbering{arabic}

\begin{center}
{\LARGE{\textbf{Computational Theories of Collaboration}}} \\
\Large\textsc{Ph.D. Comprehensive Exam} \\[1em]
\large\textnormal{Mohammad Shayganfar - mshayganfar@wpi.edu} \\
\large\textnormal{May, 26 2015}
\end{center}

\section{Introduction to Collaboration Theories}

To be collaborative, partners, e.g., a robot and a human, need to meet the
specifications stipulated by some theories that we review in this document. As
we discuss in Section \ref{sec:sharedplans}, collborators need to commit to the
group activity and to their role in it; they need to divide the task load
according to their capabilities so they can carry out the individual plans that
constitute the group activity; and they need to commit to the success of others.
Collaborators also need to be able to communicate with others effectively, and
to interpret others' actions and utterances in the collaboration context.
Furthermore, collaborators need to be willing to help others in doing their own
tasks, and to reconcile between commitments to existing collabortion and their
other activities \cite{grosz:mice-menus}.

Collaboration is a special type of coordinated activity in which the
participants work jointly, together performing a task or carrying out the
activities needed to satisfy a shared goal \cite{grosz:collaboration}.

Existing collaboration theories (including SharedPlans) consider the nature of a
collaboration to be more than a set of individual acts. These theories argue for
an essential distinction between a collaboration and a simple interaction or
even a coordination in terms of commitments \cite{grosz:shared-plans,
lochbaum:collaborative-planning}.

\section{Teamwork \& Collaboration}

Bratman's view is that collective intention can be described by referring to
individual intention in combination with other mental attitudes. Searle's
opposing view is that collective intention cannot be so reduced.

\section{Computational Theories of Collaboration}

There are prominent collaboration theories that are mostly based on plans and
often analysis of the discourse between collaborators revolving around these
plans \cite{grosz:plans-discourse, Litman:discourse-commonsense}. In these
theories the discourse analysis is based on search over these tree plans
\cite{rich:discourse}.

The two theories Joint Intentions and SharedPlans have been extensively used to
examine and describe teamwork.

Two important theories for modeling teamwork collaboration were derived from
the BDI paradigm.

\subsection{Theory of Joint Intentions/Teamwork}

The Joint Intentions Framework \cite{levesque:acting-together}
\cite{cohen:intention-commitment} \cite{cohen:persistence-intention-commitment}
is a theoretical framework founded on BDI logics. The framework focuses on a
team's joint mental state, called a joint intention. A team jointly intends a
team action if all team members are jointly committed to perform an action while
in a specified mental state.

Joint Intentions Theory describes how a team of agents can jointly act together
by sharing mental states about their actions. An intention is viewed as a
commitment to perform an action while in a mental state. And a joint intention
is a shared commitment to perform an action while in a group mental state
\cite{cohen:intention-commitment}. Communication is required to establish and
maintain mutual beliefs and joint intentions. A team of agents jointly intend to
perform an action if and only if the members have a joint persistent goal
\cite{cohen:teamwork}.

In order to enter a joint commitment, the team members have to establish
appropriate mutual beliefs and individual commitments. Although the Joint
Intentions Theory does not mandate communication and several techniques are
available to establish mutual beliefs about actions from observations (see
\cite{huber:withour-communication}), currently communication seems to be the
only feasible way to attain joint commitments. A key aspect of the Joint
Intention Theory is the commitment to attain mutual belief about the termination
of a team action. This helps to ensure that the team stays updated about the
status of the team actions. This behaviour is achieved by enforcing that agents
committing to a joint intention also commit to inform their team about any
relevant failures or premature terminations. Joint intentions and joint
commitments provide a basic framework to reason about coordination required for
teamwork as well as guidance for monitoring and maintaining team activities.
However, a single joint intention for a high-level team goal is not sufficient
to model team behaviour in detail and to ensure coherent teamwork.

There are also some other theories with similarities and contrasts conveying
collaboration concepts including Cohen and Levesque's work describing the
concept of \textit{joint intentions} in \cite{cohen:teamwork,
levesque:acting-together}.

In \cite{cohen:teamwork} Cohen and Levesque establish that joint intention
cannot be defined simply as individual intention with the team regarded as an
individual. This is because after the initial formation of an intention, team
members may diverge in their beliefs and hence in their attitudes towards the
intention. Instead, Cohen and Levesque generalise their own definition of
intention. First they present a definition of individual persistent goal and, in
terms of this, individual intention. Both definitions use the notion of
individual belief. Next, they define precise analogues of these concepts --
joint persistent goal and joint intention -- by invoking mutual belief in place
of individual belief. The definition of joint persistent goal additionally
requires each team member to commit to informing other members -- to the extent
of the team's mutual belief -- if it comes to believe that the common goal has
been achieved, becomes impossible or is no longer relevant. The result is that,
while a team is not an individual, nevertheless joint intention is similar to
individual intention. In Cohen and Levesque's theory, then, a team with a joint
intention is a group that shares a common objective and a certain shared mental
state. In particular, joint intentions are held by the team as a whole
\cite{jarvis:teams-multiagent-systems}.

\subsection{SharedPlans Theory}
\label{sec:sharedplans}

\subsubsection{Communicating Intentions}

Using discourse plans can help to encode the knowledge about conversation.

The SharedPlans theory recognises three interrelated levels of discourse
structure.

In \cite{grosz:plans-discourse}, Grosz and Sidner argue that the components of
the discourse structure are a trichotomy of linguistic structure, intentions
structure and the attention state. In their work, the linguistic structure of a
discourse is a sequence of utterances aggregating into discourse segments just
as the words in a single sentence form constituent phrases. They also discuss
the idea of the discourse purpose as the intention that underlies engagement in
the particular discourse. They believe this intention is the reason behind
performing a discourse rather than some other actions, and also the reason
behind conveying a particular content of the discourse rather than some other
contents. They describe mechanisms for plan analysis looking at Discourse
Segment Purposes (DSPs). In fact, the DSPs specify how the discourse segments
contribute to achieving the overall discourse purpose. Finally, the third
component in their theory, the attentional state, provides an abstraction of the
agent's focus of attention as the discourse unfolds. The focusing structure
contains DSPs and the stacking of focus spaces reflects the relative salience of
the entities in each space during the discourse. In short, the focusing
structure is the central repository for the contextual content required for
processing utterances during the discourse \cite{grosz:plans-discourse}.

\subsubsection{Collaboration Vs. Sum of Coordinate Actions}

\cite{grosz:collaborative-systems}

\subsubsection{SharedPlans}

Grosz and Sidner \cite{grosz:plans-discourse}
Grosz and Kraus \cite{grosz:collaboration}

SharedPlans is a general theory of collaborative planning that requires no
notion of joint intentions, accommodates multi-level action decomposition
hierarchies and allows the process of expanding and elaborating partial plans
into full plans.

SharedPlans is rooted in the observation that collaborative plans are not simply
a collection of individual plans, but rather a tight interleaving of mutual
beliefs and intentions of different team members.

The SharedPlans model of collaborative action \cite{grosz:planning-acting}
\cite{grosz:collaboration} \cite{grosz:plans-discourse} aims to provide the
theoretical foundations needed for building collaborative robots/ agents
\cite{grosz:collaborative-systems}. It specifies four key characteristics for
participants in a group activity to be collaborative partners, and thus for
their joint activity to be collaborative. The SharedPlans definition states that
for a group activity to be collaborative, the collaborators must have:

Look at this carefully!!!

\begin{enumerate}[a)]
  \item individual intentions that the group perform the group activity;
  \item mutual belief of a (partial or complete) recipe;
  \item individual or group plans for the constituent subactions of the recipe;
  \item intentions that their collaborators succeed in doing the constituent
  subactions.
\end{enumerate}
 
In other words, to successfully complete a plan the collaborators must mutually
believe that they have a common goal and have agreed on a sequence of actions
for achieving that goal. They should believe that they are both capable of
performing their own actions and intend to perform those actions while they are
committed to the success of their plans.

The intentions that this definition specifies constitute different kinds of
commitments required of the collaborators. 

The idea behind partial shared plans is enabling the agents to modify the shared
plan over the course of planning without impairing the achievement of the shared
goals.

\subsubsection{Intention-to and Intention-that}

In Grosz and Sidner's SharedPlans theory \cite{grosz:plans-discourse}, two
intentional attitudes are employed: ``intending to" (do an action) and
``intending that" (a proposition will hold). The notion of \textit{intention
to}, as an individual-oriented intention, models the intention of an agent to do
any single-agent action while the agent not only believes that it is able to
execute that action, but it also committs to doing so. In contrast with
\textit{intention to}, an \textit{intention that}, as the notion of an intention
directed toward group activity, does not directly imply an action. In fact, an
individual agent's \textit{intention that} is directed towards its
collaborator's action or towards a group's joint action. \textit{Intention that}
guides an agent to take actions (including the communication), that enable or
facilitate other collaborators to perform assigned tasks. This leads an agent to
behave in a manner consistent with a collaborative effort. Therefore, agents
will adopt intentions to communicate about the plan \cite{grosz:collaboration}.

\subsubsection{Recipes}

The SharedPlans definition of mutual beliefs states that when agents have a
shared plan for doing some act, they must hold mutual beliefs about the way in
which to perform that act. Following Pollack
\cite{pollack:plan-mental-attitudes}, the term recipe refers to what
collaborators know when they know a way of doing something. Recipes are
specified at a particular level of detail. Hence, the agents need to have mutual
beliefs about acts specified at the particular level of detail of the recipe,
and they do not to have mutual beliefs about all levels of acts that each agent
will perform. Mutual belief of the recipe essentially means that all the
collaborators hold the same beliefs about the way in which the activity will be
accomplished. Therefore, the collaborators must agree on how to do the activity.
Grosaz and Sidner in their earlier work \cite{grosz:plans-discourse} have
considered only simple recipes in which each recipe consisted of only a single
act-type relation \cite{lochbaum:plan-models}. Recipes are aggregations of
act-types and relations among them. Act-types, rather than actions, are the main
elements in recipes. Recipes can be partial, meaning thay can expand and be
modified over time.

Finally, the definition of the overall plan in terms of constituent plans of
individuals or groups is recursive, with the recursion ending at the level of
basic, individual actions \cite{grosz:mice-menus}.

Grosz, Sidner and Lochbaum in \cite{grosz:plans-discourse} and
\cite{lochbaum:plan-models} present a model of plans to account for how agents
with partial knowledge collaborate in the construction of a domain plan. Agents
have a library of partially speci ed plan schemas (recipes). These recipes might
be underspeci ed as to how an action is executed or how an action contributes to
a goal. Agents then collaborate in constructing a shared plan by uttering
statements about their beliefs and intentions about the plan. This collaboration
will terminate with each agent mutually believing that each act in the plan can
be executed by one of the agents, that that agent intends to perform the act,
and that each act in the plan contributes to the goal.

Grosz, Sidner and Lochbaum in \cite{grosz:plans-discourse} and
\cite{lochbaum:plan-models} are interested in the type of plans that underlie
discourse in which the agents are collaborating in order to achieve a shared
goal. They propose that agents are building a shared plan in which participants
have a collection of beliefs and intentions about the actions in the plan.

Grosz, Sidner and Lochbaum in \cite{grosz:plans-discourse} and
\cite{lochbaum:plan-models} model how several agents with partial knowledge
collaborate on constructing a shared domain plan . Each agent communicates their
beliefs and intentions by making utterances about what actions they can
contribute to the shared plan. Collaboration is again modelled by the agents
establishing a mutual belief that each action in the shared plan contributes to
the goal of the plan, and that each action can and will be performed by one of
the agents.

\textit{Shared plan} is another essential concept in the collaboration context.
The definition of the shared plan is derived from the definition of plans
Pollack introduced in \cite{pollack:plan-inference,
pollack:plan-mental-attitudes} since it rests on a detailed treatment of the
relations among actions and it distinguishes the intentions and beliefs of an
agent about those actions. However, since Pollack's plan model is just a simple
plan of a single agent, Grosz and Sidner extended that to plans of two or more
collaborative agents. The concept of the shared plan provides a framework in
which to further evaluate and explore the roles that particular beliefs and
intentions play in collaborative activity \cite{lochbaum:plan-models}. However,
this formulation of shared plans (a) could only deal with activities that
directly decomposed into single-agent actions, (b) did not address the
requirement for the commitment of the agents to their joint activities, and (c)
did not adequately deal with agents having partial recipes
\cite{grosz:collaboration}. Grosz and Kraus in \cite{grosz:collaboration},
reformulate Pollack's definition of the individual plans
\cite{pollack:plan-mental-attitudes}, and also revise and expand the SharedPlans
to address these shortcomings.

\subsubsection{Coordinated Cultivation of SharedPlans}

Grosz and Hunsberger \cite{grosz:ccsp} claim to reconcile the two approaches.
They provide the ``Coordinated Cultivation of SharedPlans" (CCSP) model, which,
while relying solely on individual intention, captures the essential properties
argued for in accounts that require group-oriented intention. CCSP also provides
a general architecture for collaboration-capable agents.

\subsection{Hybrid Collaboration Approaches}

Tambe's work on \textit{STEAM teamwork model} \cite{tambe:flexible-teamwork}.

Jennings work in \cite{jennings:joint-intention-hybrid}

STEAM (Shell for Teamwork) builds on both Joint Intention Theory and Shared Plan
Theory and tries to overcome their shortcomings. Based on joint intentions,
STEAM builds up hierarchical structures that parallel the Shared Plan Theory as
described in the previous chapter. Hence, STEAM formalises commitments by
building an  maintaining Joint Intentions, and uses SharedPlans to formulate the
team's attitudes in complex tasks.

In \cite{tambe:flexible-teamwork} Tambe presents STEAM, an implemented model of
teamwork based primarily on Cohen et al.'s theory of Joint Intentions, but
informed by key concepts from SharedPlans. Following Cohen et al., a team
initially adopts a joint intention for a high-level team goal that includes
commitments to maintain the goal until it is deemed already achieved,
unachievable or irrelevant. The agents then construct a hierarchy of individual
and joint intentions analogous to partial SharedPlans. Tambe notes that as the
hierarchy evolves, if a step involves only a subteam then that subteam must form
a joint intention to perform that step, and the remaining team members need only
track the subteam's joint intention, requiring that they be able to infer
whether or not the subteam intends to, or is able to, execute that step
\cite{hunsberger:shared-plans-easier}.

\section{Relation to Psychology and Sociology}

Referring expressions \cite{heeman:model-collaboration-referring}

\section{Similarities and Differences}

\begin{enumerate}
  \item None of SharedPlans' four components (see Section \ref{sec:sharedplans})
  has the notion of a joint intention. This is a significant difference between
  SharedPlans and Joint Intentions theories, since the notion of joint intention
  is an integral part of Cohen and Levesque's theory. In particular, SharedPLans
  theories emphesizes on the agents individually intending that the joint action
  be done successfully as well as the agents individually intending the success
  of their collaborators' actions which is introduced in
  \cite{grosz:collaboration} by Grosz and Kraus as the notion of
  \textit{intention-that}.
  \item Communication requirements are derived from any intention that's, as
  opposed to being ``hard-wired" in Joint Intentions.
  \item In contrast to Joint Intentions, the SharedPlans Theory employs
  hierarchical structures over intentions, thus overcoming the shortcoming of a
  single Joint Intention for complex team tasks. The Shared Plans Theory is not
  based on a joint mental attitude but on an intentional attitude called
  intending that, which is very similar to an agent’s normal intention to
  perform an action.
  \item Another main difference between the Joint Intentions Theory and the
  SharedPlans Theory is that the Shared Plans Theory describes the way to
  achieve a common goal through the hierarchy of plans, whereas the Joint
  Intentions Theory describes only this common goal
  \cite{skubch:modelling-behavior-robots}.
  \item Joint Intentions theory assumes that knowledge about the team-mates is
  always available.
\end{enumerate}

\section{Application in Human-Computer Collaboration}

COLLAGEN \cite{rich:collaboration-manager,rich:discourse} is the first
implemented system based on the SharedPlans theory. It incorporates certain
algorithms for discourse generation and interpretation, and is able to maintain
a segmented interaction history, which facilitates the discourse between human
user and the intelligent agent. The model includes two main parts: (1) a
representation of discourse state and (2) a discourse interpretation algorithm
utterances of the user and agent \cite{rickel:discourse-theory-dialogue}.

In \cite{heeman:model-collaboration-referring} Heeman presents a computational
model of how a conversational participant collaborates in order to make a
referring action successful. The model is based on the view of language as
goal-directed behaviour, and in his work, he refers to SharedPlans as part of
the planning and conversation literature.

In \cite{lochbaum:plan-models}, Lochbaum and Sidner modify and expand the
SharedPlan model of collaborative behavior \cite{grosz:plans-discourse}. They
present an algorithm for updating an agent’s beliefs about a partial SharedPlan
and describe an initial implementation of this algorithm in the domain of
network management.

The system GRATE* by Jennings \cite{jennings:joint-intention-hybrid} is based on
the Joint Intention Theory. GRATE* provides a rule-based modelling approach to
cooperation using the notion of Joint Responsibilities, which in turn is based
on Join Intentions. GRATE* is geared towards industrial settings in which both
agents and the communication between them can be considered to be reliable.

CAST (Collaborative Agents for Simulating Teamwork) \cite{yen:cast}
\cite{yin:knowledge-based-sharedplans} is a teamwork framework based on the
SharedPlans Theory. CAST focuses on flexibility in dynamic environments and on
proactive information exchange enabled by anticipating what information team
members will need. Petri Nets are used to represent both the team structure and
the teamwork process, i.e., the plans to be executed.

There are many research focusing
on different aspects of collaboration each of which are different than my own
work. In my thesis, I focus on emotion functions and how they impact
collaboration's structure and processes, and how the dynamics of the
collaboration structure influences emotion-regulated processes. Some of the
other works focus on the concepts of robot assistants
\cite{clancey:agent-assistants-collaboration}, or teamwork and its challenges in
cognitive and behavioral levels \cite{cohen:teamwork,
nikolaidis:collaboration-joint-action, scerri:prototype-distributed-teams,
tambe:flexible-teamwork}. Some researchers have an overall look at a
collaboration concept at the architectural level. In
\cite{garcia:collaboration-emotional-awareness} authors present a collaborative
architecture, COCHI, to support the concept of emotional awareness. In
\cite{esau:integrating-emotion-collaboration} authors present the integration of
emotional competence into a cognitive architecture which runs on a robot, MEXI.
In \cite{sofge:collaboration-humanoid-space} authors discuss the challenges of
integrating natural language, gesture understanding and spatial reasoning of a
collaborative humanoid robot situated in the space. The importance of
communication during collaboration has been considered by some researchers from
human-computer interaction and human-robot collaboration
\cite{clair:action-intention-collaboraiton,
matignon:verbal-nonverbal-collaboration, rich:discourse} to theories describing
collaborative negotiation, and discourse planning and structures
\cite{andriessen:disourse-planning, grosz:discourse-structure,
sidner:discourse-collaborative-negotiation}. There are other concepts such as
joint actions and commitments \cite{grosz:intention-dynamics-collaboration},
dynamics of intentions during collaboration \cite{levesque:acting-together}, and
task-based planning providing more depth in the context of collaboration
\cite{burghart:cognitive-architecture-robot, rich:cea}.

The concept of collaboration has also received attention in the industry and in
research in robotic laboratories \cite{green:collaboration-literature-review}.

\section{Conclusion}

\bibliographystyle{plain}
\bibliography{mshayganfar}

\end{document}
