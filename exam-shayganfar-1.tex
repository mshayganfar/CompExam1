\documentclass[11pt]{article}

\usepackage{graphicx}
%\usepackage{algorithmic}
%\usepackage{algorithm}
\usepackage{amssymb}
\usepackage{amsmath}
\usepackage[mathscr]{euscript}

\begin{document}

\pagenumbering{arabic}

\begin{center}
{\LARGE{\textbf{Computational Theories of Collaboration}}} \\
\Large\textsc{Ph.D. Comprehensive Exam} \\[1em]
\large\textnormal{Mohammad Shayganfar - mshayganfar@wpi.edu} \\
\large\textnormal{May, 26 2015}
\end{center}

\section{Introduction to Collaboration Theories}

Collaboration is a special type of coordinated activity in which the
participants work jointly, together performing a task or carrying out the
activities needed to satisfy a shared goal \cite{grosz:collaboration}.

Existing collaboration theories (including SharedPlans) consider the nature of a
collaboration to be more than a set of individual acts. These theories argue for
an essential distinction between a collaboration and a simple interaction or
even a coordination in terms of commitments \cite{grosz:shared-plans,
lochbaum:collaborative-planning}.

\section{Computational Theories of Collaboration}

There are prominent collaboration theories that are mostly based on plans and
often analysis of the discourse between collaborators revolving around these
plans \cite{grosz:plans-discourse, Litman:discourse-commonsense}. In these
theories the discourse analysis is based on search over these tree plans
\cite{rich:discourse}.

\subsection{SharedPlans Theory}

In \cite{grosz:plans-discourse}, Grosz and Sidner argue
that the components of the discourse structure are a trichotomy of linguistic
structure, intentions structure and the attention state. In their work, the
linguistic structure of a discourse is a sequence of utterances aggregating into
discourse segments just as the words in a single sentence form constituent
phrases. They also discuss the idea of the discourse purpose as the intention
that underlies engagement in the particular discourse. They believe this
intention is the reason behind performing a discourse rather than some other
actions, and also the reason behind conveying a particular content of the
discourse rather than some other contents. They describe mechanisms for plan
analysis looking at Discourse Segment Purposes (DSPs). In fact, the DSPs
specify how the discourse segments contribute to achieving the overall discourse
purpose. Finally, the third component in their theory, the attentional state,
provides an abstraction of the agent's focus of attention as the discourse
unfolds. The focusing structure contains DSPs and the stacking of focus spaces
reflects the relative salience of the entities in each space during the
discourse. In short, the focusing structure is the central repository for the
contextual content required for processing utterances during the discourse
\cite{grosz:plans-discourse}.

\textit{Shared plan} is another essential concept in the collaboration context.
The definition of the shared plan is derived from the definition of plans
Pollack introduced in \cite{pollack:plan-inference,
pollack:plan-mental-attitudes} since it rests on a detailed treatment of the
relations among actions and it distinguishes the intentions and beliefs of an
agent about those actions. However, since Pollack's plan model is just a simple
plan of a single agent, Grosz and Sidner extended that to plans of two or more
collaborative agents. The concept of the shared plan provides a framework in
which to further evaluate and explore the roles that particular beliefs and
intentions play in collaborative activity \cite{lochbaum:plan-models}. However,
this formulation of shared plans (a) could only deal with activities that
directly decomposed into single-agent actions, (b) did not address the
requirement for the commitment of the agents to their joint activities, and (c)
did not adequately deal with agents having partial recipes
\cite{grosz:collaboration}. Grosz and Kraus in \cite{grosz:collaboration},
reformulate Pollack's definition of the individual plans
\cite{pollack:plan-mental-attitudes}, and also revise and expand the shared plan
to address these shortcomings.

\subsection{Theory of Joint Intentions}

There are also some other theories with similarities and contrasts conveying
collaboration concepts including Cohen and Levesque's work describing the
concept of \textit{joint intentions} in \cite{cohen:teamwork,
levesque:acting-together}.

\subsection{``Hybrid" Collaboration Theories}

Tambe's work on \textit{STEAM teamwork model} \cite{tambe:flexible-teamwork}.

\section{Relation to Psychology and Sociology}

\section{Similarities and Differences}

\section{Application in Human-Computer Collaboration}

There are many research focusing
on different aspects of collaboration each of which are different than my own
work. In my thesis, I focus on emotion functions and how they impact
collaboration's structure and processes, and how the dynamics of the
collaboration structure influences emotion-regulated processes. Some of the
other works focus on the concepts of robot assistants
\cite{clancey:agent-assistants-collaboration}, or teamwork and its challenges in
cognitive and behavioral levels \cite{cohen:teamwork,
nikolaidis:collaboration-joint-action, scerri:prototype-distributed-teams,
tambe:flexible-teamwork}. Some researchers have an overall look at a
collaboration concept at the architectural level. In
\cite{garcia:collaboration-emotional-awareness} authors present a collaborative
architecture, COCHI, to support the concept of emotional awareness. In
\cite{esau:integrating-emotion-collaboration} authors present the integration of
emotional competence into a cognitive architecture which runs on a robot, MEXI.
In \cite{sofge:collaboration-humanoid-space} authors discuss the challenges of
integrating natural language, gesture understanding and spatial reasoning of a
collaborative humanoid robot situated in the space. The importance of
communication during collaboration has been considered by some researchers from
human-computer interaction and human-robot collaboration
\cite{clair:action-intention-collaboraiton,
matignon:verbal-nonverbal-collaboration, rich:discourse} to theories describing
collaborative negotiation, and discourse planning and structures
\cite{andriessen:disourse-planning, grosz:discourse-structure,
sidner:discourse-collaborative-negotiation}. There are other concepts such as
joint actions and commitments \cite{grosz:intention-dynamics-collaboration},
dynamics of intentions during collaboration \cite{levesque:acting-together}, and
task-based planning providing more depth in the context of collaboration
\cite{burghart:cognitive-architecture-robot, rich:cea}.

The concept of collaboration has also received attention in the industry and in
research in robotic laboratories \cite{green:collaboration-literature-review}.

\section{Conclusion}

\bibliographystyle{plain}
\bibliography{mshayganfar}

\end{document}
